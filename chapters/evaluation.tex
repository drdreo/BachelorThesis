\chapter{Evaluierung}
In diesem Kapitel wird objektiv die Webkomponenten Technologie evaluiert.

\section{Nutzen}
Durch die Inkompatibilität der Webkomponenten mit diversen Browsern, können diese für manche EntwicklerInnen noch nicht als Produktionstechnologie verwendet werden. Jedoch haben bereits namhafte Firmen -- YouTube, GitHub, \etc -- Webkomponenten im Einsatz \cite{webcomponents-production-use}. 
Da Webkomponenten noch nicht richtig -- in der Industrie -- in Verwendung sind, gibt es noch eine überschaubare Gemeinschaft die hinter den Webkomponenten steht. 

Wurde die Technologie erstmals durchschaut, kann durch das Wiederverwenden der erstellten Komponenten Zeit gespart, und so die Produktivität gesteigert werden. Jedoch sollte bedacht werden, dass der Funktionsumfang, welchen ein Framework bietet, die gleichen und darüber hinaus noch weitere Funktionen zur Verfügung stellen kann. Alle Ziele, die mit Webkomponenten erreicht werden können, können gleichermaßen auch mit einem Framework erzielt werden. Lediglich auf eine andere Art und Weise. 

\section{Flexibilität}
Der Vorteil der Webkomponenten liegt darin, sie nach Belieben einsetzen zu können. Damit ist gemeint, dass in bestehenden Projekten Teile durch Webkomponenten ausgetauscht und nachgebessert werden können. Egal, ob es ein "`vanilla"' Projekt, oder ein mit beispielsweise Angular aufgesetztes Projekt, ist. Bei einer Anwendung, welche nicht mit dem gewünschten Framework erstellt wurde, können meist keine Elemente dieses Frameworks genutzt werden. Bei Webkomponenten fällt dieser Nachteil weg.

\section{Wiederverwendbarkeit}

Die Abkapselung der Webkomponenten vom restlichen Dokument und prinzipiell auch dem Projekt selbst, macht diese wiederverwendbar. Hat man eine Komponente für eine  bestimmte Anwendung programmiert, so ist diese nicht zwingend an diese gebunden und kann auch in künftigen, oder vergangenen Projekten, verwendet werden. Durch ihre Flexibilität wird auch die Wiederverwendbarkeit gesteigert, da man Webkomponenten auch in Projekten, welche auf einem Framework aufbauen, wieder verwenden kann. So wird das endlose Reimplementieren der selben Funktionen in unterschiedlichen Projekten obsolet. Ebenso ist es möglich Komponenten anderen EntwicklerInnen zur Verfügung zu stellen, sodass man eine große Palette an bereits vorhandenen Komponenten hat, um Anwendungen zu realisieren.

Das Wiederverwenden ist möglich, da Webkomponenten auf derzeitigen oder geplanten Webstandards basieren, wie in Kap.~\ref{cha:webcomponents} erwähnt, welche alle größeren Browser versuchen zu implementieren. Das bedeutet, dass Webkomponenten nicht auf ein Framework oder eine Bibliothek angewiesen sind, und universell im Web funktionieren können.

\section{Testbarkeit}

Da Komponenten abgeschlossene Systeme sind, können sie als Blackbox betrachtet werden. Diese sind von deren Umgebung komplett unabhängig. Diese Unabhängigkeit bringt den Idealfall für Programmtests mit sich, da es stets bequemer ist, abgeschottete Einheiten zu testen, als den ganzen Komplex.

Scram.js hat eine Option(-d), die es ermöglicht, ein Electron Fenster während des Debuggens zu öffnen. Dadurch hat man alle Chrome-Entwicklerwerkzeuge für das Debuggen am Server zur Verfügung.

Um Webkomponenten zu testen gibt es vom Polymer-Team den sogenannten "`Web Component Tester"'\footnote{https://www.polymer-project.org/1.0/docs/tools/tests}. Dieser ist eine End-to-End Testumgebung, welche das lokale Testen von selbst erstellten Elementen ermöglicht.