\chapter{Evaluierung}
In diesem Kapitel wird objektiv die Webkomponenten Technologie evaluiert.

\section{Nutzen}
Durch die Inkompatibilität der Webkomponenten mit diversen Browsern, können diese für manche EntwicklerInnen noch nicht als Produktionstechnologie verwendet werden. Jedoch haben schon einige namhafte Firmen -- YouTube, GitHub, \etc -- Webkomponenten im Einsatz \cite{webcomponents-production-use}. 
Da sie noch nicht richtig -- in der Industrie -- in Verwendung sind, gibt es noch eine überschaubare Gemeinschaft die hinter den Webkomponenten steht. 

Hat man erstmal die Technologie durchschaut kann durch das Wiederverwenden der erstellten Komponenten Zeit gespart, und so die Produktivität gesteigert werden. Jedoch sollte man bedenken, dass der Funktionsumfang, welcher ein Framework bietet, die gleichen und auch mehr Funktionen zur Verfügung stellen kann. Alles was Webkomponenten erreichen können, kann ebenso mit einem Framework erreicht werden. Lediglich auf eine andere Art und Weise. 
%TODO Weiterführen, wann man webkomponenten nehmen sollte und wann nicht

\section{Flexibilität}
Der Vorteil der Webkomponenten liegt darin, sie nach belieben einsetzen zu können. Damit ist gemeint, man kann in bestehenden Projekten Teile durch Webkomponenten austauschen und nachbessern. Egal, ob es ein "`vanilla"' Projekt, oder ein mit beispielsweise Angular aufgesetztes Projekt, ist. Bei einer Anwendung, welche nicht mit dem gewünschten Framework erstellt wurde, kann man meist keine Elemente dieses Frameworks nutzen. Bei Webkomponenten fällt dieser Nachteil weg.

\section{Wiederverwendbarkeit}
Die Abkapselung der Webkomponenten vom restlichen Dokument und prinzipiell auch dem Projekt selbst, macht diese sehr wiederverwendbar. Hat man eine Komponente für eine Anwendung programmiert, so ist diese nicht zwingend an diese gebunden und kann auch in künftigen, oder vergangenen Projekten, verwendet werden. Ebenso ist es möglich Komponenten anderen EntwicklerInnen zur Verfügung zu stellen, sodass man eine große Palette an bereits vorhandenen Komponenten hat, um Anwendungen zu realisieren. Durch ihre Flexibilität wird auch die Wiederverwendbarkeit gesteigert, da man Webkomponenten auch in Projekten, welche auf einem Framework aufbauen, wieder verwenden kann. 

\section{Testbarkeit}
Um Webkomponenten zu testen gibt es vom Polymer-Team den sogenannten "`Web Component Tester"'\footnote{https://www.polymer-project.org/1.0/docs/tools/tests}. Dieser ist eine End-to-End Testumgebung welcher das lokale Testen von selbst erstellten Elementen ermöglicht.