\chapter{Einleitung}

\section{Motivation}

Die Idee, die Webprogrammierung komponentenorientiert zu gestalten, gibt es seit 1998 \cite{microsoft-webcomponents}. Jedoch hat sich diese Idee, bis heute, nicht zu einem Web-Standard weiterentwickelt.
Seit einigen Jahren gibt es dazu wieder Ansätze, welche von größeren Firmen, wie beispielsweise Google mit Angular\footnote{https://angular.io/} oder Facebook mit React\footnote{https://facebook.github.io/react/}, mittels eigens entwickelten komponentenbasierten Frameworks versuchen, die komponentenorientierte Webentwicklung voran zu treiben. 
Allerdings ist seit neuestem die grundlegende Idee der nativen Webkomponenten\footnote{https://www.webcomponents.org/}, welche künftig nur als Webkomponenten bezeichnet werden, zurück. Da diese, sobald gängige Webbrowser alle Technologien von Webkomponenten -- HTML Imports, Templates, Custom Elements, Shadow DOM -- unterstützen, einen weiteren Schritt in standardisierte, wiederverwendbare und unabhängige Webbausteine bedeutet. Zur Zeit ist es noch nicht, oder erschwert möglich, clientseitig und  serverseitig auf Webkomponenten zu setzen.
Diese Bachelorarbeit beleuchtet jene Problematik und stellt Lösungsvorschläge anhand einer konkreten CRUD-Anwendung dar.


\section{Forschungsfrage}
\label{sec:hintergrund}

Eine CRUD-Anwendung (Create, Read, Update, Delete) ist eine Anwendung, welche die vier grundlegenden Operationen implementiert, um Daten zu manipulieren. BenutzerInnen können durch eine Benutzeroberfläche Datensätze erstellen, lesen, bearbeiten und löschen. 
Angular oder React wären für die Umsetzung einer solchen Anwendung geeignet. Aus heutiger Sicht zählen diese Technologien zu den gängigsten Frameworks und werden von zahlreichen EntwicklerInnen stetig verbessert.
Diese besagten Frameworks sind mit einer steilen Lernkurve behaftet und benötigen Zeit um sich damit vertraut zu machen.

Im Gegensatz dazu sind Webkomponenten logisch deklariert und bauen lediglich auf Kenntnissen von nativem HTML, CSS und Javascript auf. Die Unterstützung wird noch nicht von jedem Browser gewährleistet und verursacht aus diesem Grund ein Zögern bei einigen EntwicklerInnen, um komplett auf Webkomponenten zu setzen.
Google hat mit Polymer\footnote{https://www.polymer-project.org/} eine Javascript-Bibliothek geschaffen, welche die Kreierung und Nutzung von Webkomponenten ungemein vereinfacht. Darüber hinaus nutzt diese Bibliothek Polyfills, die Funktionen, welche Polymer benötigt, bei Bedarf liefert und wird dadurch nicht nur in allen gängigen Webbrowsern -- Chrome, Firefox, Edge, Safari --  sondern auch in deren älteren Versionen, die die notwendigen Funktionen nicht implementiert haben, unterstützt \cite{polymer-compatibility}. Native Unterstützung ist bereits bei den meisten Browserherstellern in Entwicklung und würde Polyfills überflüssig machen.\newline
\newline Aus diesen Ansätzen ergibt sich folgende Forschungsfrage für diese Bachelorarbeit:

\begin{quote}
Wie muss eine Webanwendung programmatisch konzipiert werden, um die grundlegenden Funktionen weitgehend mittels server- und clientseitigen Webkomponenten zu realisieren?
\end{quote}

\section{Methodik}

Um diese Frage zu beantworten, soll die Bachelorarbeit als eine Kombination von Literaturarbeit und praktischer \bzw prototypischer Umsetzung realisiert werden.

Zunächst soll aus bestehender Literatur (erweiternd zu Abschnitt \ref{sec:hintergrund}) erörtert werden, wie mit dem Thema der Webkomponenten und mit komponentenorientierten Frameworks umgegangen wird. Gemeinsame Faktoren sollen daraus extrahiert werden und als Grundlage für ein eigenes, theoretisches Konzept dienen. Die Vorteile und Limitierungen von Webkomponenten werden dabei erörtert. Dieses Konzept soll schlussendlich eine Liste von Kerntechniken enthalten, welche die Umsetzung einer CRUD-Anwendung mittels Webkomponenten ermöglicht.

Überprüft werden soll die Anwendbarkeit dieses Konzepts, durch einen eigenen, im Rahmen dieser Arbeit entwickelten Prototypen.

\section{Erwartete Ergebnisse}

Als konkretes Ergebnis wird ein Gerüst mit Webkomponenten erstellt, welches als Grundlage für die Erstellung einer Webanwendung dienen soll. Es wird erwartet, dass sich solche konkreten Techniken finden und umsetzen lassen.

Bei der praktischen Umsetzung der Webkomponenten in einer CRUD-Anwendung wird ebenfalls eine positive Evaluierung erwartet, da es bereits erfolgreiche Konzepte basierend auf server- und clientseitigen Webkomponenten gibt und somit auf deren Erfahrungen aufgebaut werden kann.

\section{Aufbau der Arbeit}
Im ersten Teil dieser Arbeit werden technische Grundlagen bezüglich Webkomponenten erörtert, und wie eine REST-Architektur aufgebaut ist. Im Kapitel danach werden State of the Art Technologien besprochen. Im Anschluss wird das grundlegende Konzept der praktischen Arbeit erläutert. Im fünften Kapitel wird die konkrete Implementierung der Webanwendung behandelt. Im darauffolgenden Abschnitt -- Evaluierung -- werden objektiv die Vor- und Nachteile, der im Fokus stehenden Technologie, aufgezeigt. Im siebten und letzten Kapitel wird  zusammenfassend die persönliche Meinung, bezüglich der Thematik, einfließen. 