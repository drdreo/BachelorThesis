\chapter{Implementierung der Webanwendung}

\section{Struktur der Webanwendung}
\subsection{Benutzeroberfläche}
Die Oberfläche der Anwendung beschränkt sich nur auf das Nötigste. Für die Eingabe von Daten werden lediglich Webkomponenten Inputs verwendet, welche durch die Nutzung von paper-inputs\footnote{https://www.webcomponents.org/element/PolymerElements/paper-input}, die der Materialdesignkonvention\footnote{Eine Anleitung, wie man Benutzeroberflächen und deren Komponenten gestalten sollte} folgen, passend gestaltet sind.

\subsection{Server}
Um einen Node.js Server mit dem Express Framework aufzusetzen und zu konfigurieren, wird, wie in Kap.~\ref{cha:scram-engine} erläutert, auf der Arbeit von Jordan Last aufgebaut und diese passend erweitert. 
Die Client-Server-Kommunikation erfolgt mittels Ajax-Aufrufen durch die iron-ajax Komponente\footnote{https://www.webcomponents.org/element/PolymerElements/iron-ajax}.
\subsection{Datenbank}
Da es für Datenbankzugriffe, zum Zeitpunkt dieser Arbeit, keine vorhanden Webkomponenten gibt, sind eigenes welche für alle CRUD-Operationen entwickelt und umgesetzt worden.

\subsection{Datenflow}
Der Anwender soll zu beginn Datensätze anlegen können. Im konkreten Beispiel wurde das Datenmodell eines Users mit Username, E-Mail und Passwort herangezogen.
Nachdem der Anwender die Daten für einen User ausgefüllt hat, kann dieser angelegt werden. Nach rudimentären Validierungen werden die Daten an den Server übermittelt. Dieser speichert den User in der Datenbank ab. 
In der Listenansicht sieht der Anwender alle bereits angelegten User in einer Tabelle und per Klick auf dessen Namen wird zur Detailansicht weitergeleitet. Hier wird ein Formular mit den Userdaten ausgefüllt, welche verändert und aktualisiert werden können.   
Auf der Detailseite findet sich ebenso ein Lösch-Knopf, um einen bestehenden User zu entfernen.
\section{Umsetzung mit Dependencies Electron / Scram-Engine}
\subsection{Electron}
Electron ist ein Framework, um native Cross-Plattform-Applikationen mit Web-Technologien wie JavaScript, HTML und CSS zu erstellen. Es basiert auf dem Chromium\footnote{https://www.chromium.org/} Webbrowser und Node.js.
\subsection{Scram-Engine}
\label{cha:scram-engine}
Das Scram-Engine-Projekt, erarbeitet von Jordan Last, ermöglicht es, eine HTML-Datei einem vorkonfigurierten Electronstartscript mit minimalem Aufwand, zur Verfügung zu stellen. Es wird lediglich Electron von der Kommandozeile aufgerufen und eine Startdatei mitgegeben. Die Vorkonfiguration vereinfacht das Arbeiten mit Electron durch beispielsweise das Anbieten von Kommandozeilenargumenten, um das Electronfenster zu verstecken, somit das bequeme Starten von serverseitigen Anwendungen aus der Kommandozeile heraus.
Electron ist notwendig, da es Chromium mit Node.js kombiniert. Die Scram-Engine nutzt Chromiums Fähigkeit, Webkomponenten in einen interaktiven DOM zu parsen, welcher manipuliert werden kann. Diese Webkomponenten können dadurch jeglichen Node.js Code nutzen und haben die Möglichkeit mit dem Betriebssystem zu interagieren, Datenbankaufrufe zu tätigen, prinzipiell alles, was Node.js möglich ist.
Ein reiner Node.js Server würde nicht ausreichen, da dieser HTML, Custom-Elemente oder Webkomponenten generell nicht parsen kann.
Durch die Kombination der beiden Technologien können Webkomponenten weitaus mehr leisten, als in einem Webbrowser.

Da universelle Webkomponenten auf Electron als deren Plattform angewiesen sind, muss das System, welches diese betreibt, relativ leistungsstark sein. Jordan Last arbeitet an einer Lösung, um die Electron-Abhängigkeit -- und dadurch die Chromium-Abhängigkeit -- loszuwerden. Jegliche Interessenten, an dieser Optimierung mitzuwirken, können ihn kontaktieren\footnote{https://github.com/lastmjs}.

\section{Implementierung der Komponenten}
