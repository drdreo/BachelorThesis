\chapter{Zusammenfassung}
Webkomponenten sind mächtig. Sie tragen wesentlich zu der Standardisierung der Webplattform bei. Clientseitig sieht man die Vorteile bereits und ist auf einem guten Weg, die Nutzung solcher Komponenten voran zu treiben. Ebenso gibt es Vorteile für die serverseitige Nutzung. Die Distanz zwischen client- und serverseitiger Komponenten-Nutzung wird geringer und Webkomponenten sind ein wichtiger Schritt in die richtige Richtung.

\section{Lernkurve}
Es gibt genügend EntwicklerInnen, UI/UX-DesignerInnen, AnfängerInnen und andere Personen, welche nichts mit Backend-Programmierung anfangen können, aber grundlegendes Wissen von HTML, CSS und Javascript besitzen. Durch die serverseitige Nutzung von Webkomponenten könnten solche Personen bekannte Techniken anwenden und vor allem, durch die semantische Sprache der Custom-Elemente, würden diese fähiger sein, serverseitigen Code zu erstellen und zu verstehen. Dies wäre mit dem Senken der Lernkurve gleichzusetzen.

\section{Leistung}
Die Abhängigkeit von Electron, und die daraus resultierende Leistung und Stabilität der Serverumgebung, könnten noch Probleme verursachen. Besonders wenn man die Anwendung auf ein reales Szenario skalieren möchte. Unabhängig davon, wäre meiner Meinung nach nicht die zur Verfügung stehende Leistung das Problem. Da Electron nur einen Prozess startet, welcher Node.js Code ausführt, der mehr oder weniger genau so wie ein \textit{vanilla} Node.js Prozess laufen würde. Ob die Chromium Laufzeitumgebung stabil genug ist, um Monate lang, ohne Unterbrechung, laufen zu können, wäre die wichtigere Frage. Wie es Jordan Last beschrieben hat \cite{server-side-webcomponents}.

\section{Markup}
Ebenso kann ich mir vorstellen, dass geübte ProgrammiererInnen von dem wortreichen Stil der Webkomponenten abgeschreckt werden. Es benötigt mehr Codezeilen um die selbe Aufgabe mit serverseitigen Webkomponenten zu realisieren, als mit \textit{vanilla} Javascript, aufgrund der Markup-Sprache.

\section{Prototyp}
Die Entwicklung des Prototypen, welcher alle CRUD-Operationen für die Datenmanipulation einer mongoDB-Datenbank, sowie ein Nutzermanagement mit Login-Me\-cha\-nis\-mus implementiert, gewährte einen großen Einblick in die Funktionen und Möglichkeiten der Webkomponenten. Durch das neu Erschaffen der DB-Custom Elemente wurde ersichtlich, dass aufgrund der Natur von Javascript, so einiges möglich ist. Für mich brachte die Nutzung der Webkomponenten eher eine Abkapselung von Programmcode als visuelle Schicht, als einen programmiertechnischen Vorteil. Es musste erst ein Grundgerüst geschaffen, und dann mit Funktion ausgestattet werden, welches sowieso in Javascript geschrieben wurde. 

Für geübte EntwicklerInnen würde es kaum einen Nutzen geben, Programm-Logik über Webkomponenten zu handhaben, außer die modulähnliche Arbeitsweise. Für AnfängerInnen jedoch, kann ich mir vorstellen, wäre es einfacher -- durch die Visualisierung von Code durch Komponenten -- in die Welt der Webentwicklung einzusteigen. 