%TODO Wie wird bisher serveranwendungen erstellt? 
% CMS Systeme: WordPress, TYPO3
%oder symfony, laravel
\chapter{State of the Art}
\label{cha:StateOfTheArt}
Um heutzutage eine vollständige Webanwendung -- Datenspeicher, Logik und Darstellung -- zu erstellen, gibt es unterschiedlichste Herangehensweisen. Um diesen Prozess zu vereinfachen, wird in den meisten Fällen ein Framework, wie in Kap.~\ref{cha:javascript-frameworks} erläutert, verwendet. Javascript Frameworks werden bevorzugt für das Frontend herangezogen. Das Backend kann, unabhängig vom Frontend, in jeder beliebigen Programmiersprache umgesetzt werden (Javascript, PHP, Python, ...), aber auch hier bietet die Nutzung eines Frameworks Vorteile. Um die Entwicklungstechnologien für vollständige Anwendungen kompakt zu halten, gibt es sogenannte Softwarestacks, welche Technologien zu einer Plattform vereinen und die Interoperabilität dieser gewährleisten. Mehr dazu in Kap.~\ref{cha:mean-stack}.

\section{Komponentenorientierte Javascript-Frameworks und -Bibliotheken}
\label{cha:javascript-frameworks}
Sogenannte Javascript Frameworks oder Bibliotheken -- der größte Unterschied liegt darin, dass eine Anwendung die Bibliothek einbindet und ein Framework die Anwendung an sich -- sind Werkzeuge, um den EntwicklerInnen ein Grundgerüst an Funktionalität zur Verfügung zu stellen. Ein wichtiger Aspekt ist dabei auch die Effizienz und Verwertbarkeit von erfahrenen ProgrammiererInnen entworfenen Code \cite{js-frameworks}.
Eine konkrete Definition eines Frameworks findet sich in \cite{def-framework}: 
\begin{quote}\textit{A framework is a semi-complete application. A framework provides a reusable, common structure to share among applications. Developers incorporate the framework into their own application and extend it to meet their specific needs. Frameworks differ from toolkits by providing a coherent structure, rather than a simple set of utility classes.}
\end{quote}

Im Anschluss befindet sich eine Auflistung von solchen, zu diesem Zeitpunkt populären Javascript Frameworks und Bibliotheken\footnote{\url{http://www.npmtrends.com/angular-vs-react-vs-vue-vs-@angular/core}}. Diese arbeiten alle zum Großteil mit Komponenten im Frontend.
\begin{itemize}  
	\item Angular
	\item Polymer
	\item REACT
	\item Vue.js
\end{itemize}
Alle aufgelisteten Frameworks und Bibliotheken sind verfügbar unter der MIT-Lizenz und werden im Anschluss genauer behandelt.

\subsection{Angular}
\label{sec:angular}
Angular ist ein TypeScript basierendes Framework, entwickelt und gewartet von Google, beschrieben als "`Superheroic JavaScript MVW Framework"'. Angular (auch "`Angular 2+"', "`Angular 2"' oder "`ng2"') ist der von Grund auf neu geschriebene, größtenteils inkompatible Nachfolger von AngularJS (auch "`Angular.js"' oder "`AngularJS 1.x"').
AngularJS wurde im Oktober 2010 veröffentlicht und bekommt weiterhin bug-fixes, \etc. Das neue Angular wurde im September 2016 als Version 2 veröffentlicht. Die aktuellste Version ist 4.3.6 (Stand 23. August 2017). Die Versionsnummer 3 wurde übersprungen, da eines der NPM-Pakete von Angular 2 bereits die Version v3.3.0 trug\footnote{\url{http://angularjs.blogspot.co.at/2016/12/ok-let-me-explain-its-going-to-be.html}}.

\subsubsection{Angular Komponente}
Das Kernkonzept von Angular ist die Komponente. Eine komplette Angular-Anwendung kann als Baum, bestehend aus solchen Komponenten, modelliert werden.
Die Definition laut der offiziellen Angular-Dokumentation \cite{angular-component}: 
\begin{quote}
	\begin{english}
		\textit{A component controls a patch of screen called a view. You define a component's application logic—what it does to support the view—inside a class. The class interacts with the view through an API of properties and methods.}
	\end{english}
\end{quote}

\begin{program}[!htbp]
\caption{Angular Komponente}
\begin{JsCode}
	import { Component } from '@angular/core';
	
	@Component({
		selector: 'hello-world',
		template: '<p>Hello, {{text}}!</p>',
	})
	export class HelloComponent {
		text: string;
		constructor() {
			this.text = 'World';
		}
	}
\end{JsCode}
\end{program}
Angewendet wird diese Komponente wie folgt:
\begin{JsCode}[numbers=none]
<hello-world></hello-world>
\end{JsCode}

Was folgende Ausgabe rendert:
\begin{JsCode}[numbers=none]
<p>Hello World</p>
\end{JsCode}

\subsection{Polymer }
Polymer (auch "`Polymer-Project"') ist eine Open Source Javascript-Bibliothek zur Erstellung von Webanwendungen mit Webkomponenten. Erstmals erschienen am 29. Mai 2015 und  wird -- ebenso wie Angular -- von Google entwickelt. Der letzte stabile Release(Stand 2. Februar 2018) ist die Version 2.4.0.

\subsubsection{Polymer Komponente}
Polymer baut auf der Custom-Element Spezifikation auf und fügt dem eine Reihe an Features hinzu wie beispielsweise \cite{polymer-elements}:
\begin{quote}
	\begin{itemize}
		\item Instanzmethoden für allgemeine Aufgaben
		\item Automatisierung für die Handhabung von Eigenschaften und Attributen, z. B. das Festlegen einer Eigenschaft basierend auf dem entsprechenden Attribut.
		\item Erstellen von Shadow DOM Bäumen für Elementinstanzen basierend auf einem <template>.
		\item Ein Datensystem, welches Datenbindung, Observers für Eigenschaftenänderungen und berechnete Eigenschaften unterstützt.
	\end{itemize}
\end{quote}
\begin{program}[!htbp]
\caption{Polymer Komponente}
\begin{JsCode}
	<link rel='import' href='bower_components/polymer/polymer.html'>
	<dom-module id='hello-world'>
	<template>
	<p>Hello {{text}}</p>
	</template>
	<script>
	class HelloWorld extends Polymer.Element {
		static get is() {
			return 'hello-world';
		}
		constructor() {
			super();
			this.text = 'World';
		}
	}
	customElements.define(HelloWorld.is, HelloWorld);
	</script>
	</dom-module>
\end{JsCode}
\end{program}
Angewendet wird diese Komponente wie folgt:
\begin{JsCode}[numbers=none]
<link rel='import' href='hello-world.html'>
<hello-world></hello-world>
\end{JsCode}
Was folgende Ausgabe rendert:
\begin{JsCode}[numbers=none]
<p>Hello World</p>
\end{JsCode}

\subsection{React}
React (auch "`React.js"' oder "`ReactJS"') wird als "`JavaScript Bibliothek zur Entwicklung von User-Interfaces"'\footnote{\url{https://reactjs.org/blog/2013/06/05/why-react.html}} bezeichnet. Es wurde erstmals im März 2013 von Facebook veröffentlicht -- zuvor nur firmenintern verwendet -- und wird seither entwickelt und gewartet. Die neuste Version (Stand 26. September 2017) ist 16.0.0.

\subsubsection{React Komponente}
Aus der Sicht von React sind Komponenten ident mit Javascript Funktionen. Sie können beliebig viele Inputs ("`props"' genannt) besitzen und geben React Elemente zurück, die beschreiben, was am Bildschirm ausgegeben werden soll.
Die Definition entsprechend der offiziellen React Dokumentation \cite{react-component}: 
\begin{quote}
	\begin{english}
		\textit{Components let you split the UI into independent, reusable pieces, and think about each piece in isolation.}
	\end{english}
\end{quote}

\begin{program}[!htbp]
\caption{React Komponente}
\begin{JsCode}
	import React, {Component} from "react";
	
	class HelloWorld extends Component {
		text = '';
		constructor(props) {
			super(props);
			this.text = 'World';
		}
		
		render() {
			return <p>Hello {this.text}</p>;
		}
	}
	export default HelloWorld;
\end{JsCode}
\end{program}
Angewendet wird diese Komponente wie folgt:
\begin{JsCode}[numbers=none]
import Hello from './HelloWorld';
<Hello/>
\end{JsCode}

Was folgende Ausgabe erzeugt:
\begin{JsCode}[numbers=none]
<p>Hello World</p>
\end{JsCode}

\subsection{Vue }
Vue (englische Aussprache [/vju/]), auch "`Vue.js"' genannt, beschreibt sich selbst als ein "`progressives Framework zur Entwicklung von User-Interfaces"'\footnote{\url{https://vuejs.org/v2/guide/}}. Vue wurde im Februar 2014 vom Ex-Google-Mitarbeiter Evan You veröffentlicht. Ein markantes Merkmal von Vue ist, dass es hauptsächlich von einer einzelnen Person, ohne die Stütze einer großen Firma, erschaffen wurde. Nur ein paar EntwicklerInnen arbeiten für Evan an der Weiterentwicklung des Frameworks. Die aktuellste Version ist 2.5.13 (Stand 2. Februar 2018).

\subsubsection{Vue Komponente}
Die Definition laut der Vue Dokumentation \cite{vue-component}: 
\begin{quote}
	\begin{english}
		\textit{Components are one of the most powerful features of Vue. They help you extend basic HTML elements to encapsulate reusable code. At a high level, components are custom elements that Vue’s compiler attaches behavior to. In some cases, they may also appear as a native HTML element extended with the special is attribute.}
	\end{english}
\end{quote}

\begin{program}[!htbp]
\caption{Vue Komponente}
\begin{JsCode}
	Vue.component('hello-world', {
		template: '<p>Hello {{ text }} </p>',
		data: function () {
			return {text:"World"};
		}
	})

	new Vue({
		el: '#container'
	})
\end{JsCode}
\end{program}

Angewendet wird diese Komponente wie folgt:
\begin{JsCode}[numbers=none]
<div id="container">
	<hello-world></hello-world>
</div>
\end{JsCode}

Was folgende Ausgabe erzeugt:
\begin{JsCode}[numbers=none]
<div id="container">
	<p>Hello World </p>
</div>
\end{JsCode}

\section%
{MEAN Stack%
	\protect\footnote{http://mean.io/}}%
\label{cha:mean-stack}
Der MEAN Stack ist ein Javascript Software Stack, um Webanwendung zu bauen. MEAN ist ein Akronym für MongoDB, Express.js, Angular und Node.js. 
Node.js ist eine Ausführungsumgebung um Javascript serverseitig auszuführen und als Servertechnologie zu nutzen. Die Architektur ermöglicht asynchronen Input und Output. Um Serveranwendungen vereinfacht entwickeln zu können, wird Express als Framework für Node genutzt. 
Angular dient als Frontend Framework und wurde bereits in Kap.~\ref{sec:angular} behandelt. MongoDB ist eine dokumentenorientierte NoSQL-Datenbank \cite{noSQL}. Diese speichert Daten im binären JSON Format ab und kann ebenso Javascript Code ausführen.

Durch die durchgängige Sprache wird der Datenaustausch zwischen Server und Client primitiv gehalten und erspart Kontextwechsel, welche beim Verwenden von unterschiedlichen Programmiersprachen entstehen. Da das Umdenken in den verschiedenen Sprachen wegfällt, werden nicht zwingend Spezialisten in jedem Bereich benötigt und es können so Anwendungen von weniger EntwicklerInnen realisiert werden. 
