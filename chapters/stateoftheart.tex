%TODO Wie wird bisher serveranwendungen erstellt? 
% CMS Systeme: WordPress, TYPO3
%oder symfony, laravel
\chapter{State of the Art}
\label{cha:StateOfTheArt}
Um heutzutage eine vollständige Webanwendung -- Datenspeicher, Logik und Darstellung -- zu erstellen, gibt es unterschiedlichste Herangehensweisen. Um diesen Prozess zu vereinfachen, wird in den meisten Fällen ein Framework, wie in Kap.~\ref{cha:javascript-frameworks} erläutert, verwendet. Javascript Frameworks werden bevorzugt für das Frontend herangezogen. Das Backend kann, unabhängig vom Frontend, in jeder beliebigen Programmiersprache umgesetzt werden (PHP, Python, C\#, ...), aber auch hier bietet die Nutzung eines Frameworks die selben Vorteile. Um die Entwicklungstechnologien für vollständige Anwendungen kompakt zu halten, gibt es sogenannte Softwarestacks, welche Technologien zu einer Plattform vereint und die Interoperabilität dieser gewährleistet.

\section%
{MEAN Stack%
	\protect\footnote{http://mean.io/}}%

Der MEAN Stack ist ein Javascript Software Stack, um Webanwendung zu bauen. MEAN ist ein Akronym für MongoDB, Express.js, Angular und Node.js. 
Node.js ist eine Ausführungsumgebung um Javascript serverseitig aus zu führen und als Servertechnologie zu nutzen. Die Architektur ermöglicht asynchronen Input und Output. Um Serveranwendungen vereinfacht entwickeln zu können, wird Express als Framework für Node genutzt. 
Angular dient als Frontend Framework und wurde bereits in Kap.~\ref{sec:angular} behandelt. MongoDB ist eine dokumentenorientierte NoSQL-Datenbank \cite{noSQL}. Diese speichert Daten im binären JSON Format ab und kann ebenso Javascript Code ausführen.

Durch die durchgängige Sprache wird der Datenaustausch zwischen Server und Client primitiv gehalten und erspart Kontextwechsel, welche beim Verwenden von unterschiedlichen Programmiersprachen entstehen. Da das Umdenken in den verschiedenen Sprachen wegfällt, werden nicht zwingend Spezialisten in jedem Bereich benötigt und es können so Anwendungen von weniger EntwicklerInnen realisiert werden. 

\section%
{Symfony%
	\protect\footnote{https://symfony.com/}}%

Sind zu viele Beispiele nicht überflüssig?
