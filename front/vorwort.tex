\chapter{Vorwort} 	% engl. Preface


Dies ist \textbf{Version \hgbthesisDate} der \latex-Dokumentenvorlage für 
verschiedene Abschlussarbeiten an der Fakultät für Informatik, Kommunikation
und Medien der FH Oberösterreich in Hagenberg, die mittlerweile auch 
an anderen Hochschulen im In- und Ausland gerne verwendet wird.

Während dieses Dokument anfangs ausschließlich für die Erstellung
von Diplomarbeiten gedacht war, sind nunmehr auch  
\emph{Masterarbeiten}, \emph{Bachelor\-arbeiten} und \emph{Praktikumsberichte} 
abgedeckt, wobei die Unterschiede bewusst gering gehalten wurden.

Die Verwendung dieser Vorlage ist jedermann freigestellt und an
keinerlei Erwähnung gebunden. Allerdings -- wer sie als Grundlage
seiner eigenen Arbeit verwenden möchte, sollte nicht einfach
("`ung'schaut"') darauf los werken, sondern zumindest die
wichtigsten Teile des Dokuments \emph{lesen} und nach Möglichkeit
auch beherzigen. Die Erfahrung zeigt, dass dies die Qualität der
Ergebnisse deutlich zu steigern vermag.

Der Quelltext zu diesem Dokument sowie das zugehörige
\latex-Paket sind in der jeweils aktuellen Version online
verfügbar unter
%
\begin{itemize}
\item[]\url{https://github.com/Digital-Media/HagenbergThesis}.
\end{itemize}
%
Trotz großer Mühe enthält dieses Dokument zweifellos Fehler und Unzulänglichkeiten
-- Kommentare, Verbesserungsvorschläge und passende Ergänzungen
sind daher stets willkommen, am einfachsten per E-Mail direkt an mich:
\begin{itemize}
\item[]%

Dr.\ Wilhelm Burger, Department für Digitale Medien,\newline
Fachhochschule Oberösterreich, Campus Hagenberg (Österreich)\newline
\nolinkurl{wilhelm.burger@fh-hagenberg.at}
\end{itemize}

\noindent
Übrigens, hier im Vorwort (das bei Diplom- und Masterarbeiten üblich, bei Bachelorarbeiten 
aber entbehrlich ist) kann kurz auf die Entstehung des Dokuments eingegangen werden.
Hier ist auch der Platz für allfällige Danksagungen (\zB an den Betreuer, 
den Begutachter, die Familie, den Hund, \ldots), Widmungen und philosophische 
Anmerkungen. Das sollte allerdings auch nicht übertrieben werden und sich auf 
einen Umfang von maximal zwei Seiten beschränken.




